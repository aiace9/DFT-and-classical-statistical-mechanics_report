\documentclass[12pt,notes=off,unicode]{beamer}
\usepackage[english]{babel}
\usepackage[utf8]{inputenc}
\usepackage{xcolor}
\usepackage{fancyhdr}
\usepackage{amsthm}
\usepackage{amsmath}
\usepackage{amssymb}
\usepackage{mathtools}
\usepackage{lmodern}
\usepackage{tikz}
\usepackage{siunitx}
\usepackage{eurosym}
\usepackage{braket}
\usepackage{animate}

\usepackage{listings}
\usepackage{color}
\definecolor{mygreen}{rgb}{0,0.6,0}
\definecolor{mygray}{rgb}{0.5,0.5,0.5}
\definecolor{mymauve}{rgb}
{0.58,0,0.82}

\lstset{ %
  backgroundcolor=\color{white},   % choose the background color
  basicstyle=\tiny\ttfamily,        % size of fonts used for the code
  language = Python,
  breaklines=true,                 % automatic line breaking only at whitespace
  captionpos=b,                    % sets the caption-position to bottom
  escapeinside={\%*}{*)},          % if you want to add LaTeX within your code
  stringstyle=\color{mymauve},     % string literal style
  keywordstyle=\color{blue},       % keyword style
  otherkeywords={True},
  identifierstyle=\color{red},
  %deleteidentifiers={StopIteration},
  commentstyle=\color{mygray},    % comment style
}

\usetikzlibrary{decorations}
\usetikzlibrary{er}

\graphicspath{{./img/}}

\usetheme{Frankfurt}
\setbeamercovered{dynamic}

\title{A report about DFT and classical statistical mechanics}
\author[Ruggero]{Ruggero Lot}
%\date[]{}

\institute[UNITS]{Università degli studi di Trieste}
\logo{\includegraphics[scale=.3]{img/logounits.pdf}}



\setbeamertemplate{headline}
{%
  \leavevmode%
  \begin{beamercolorbox}[wd=.5\paperwidth,ht=2.5ex,dp=1.125ex]{section in head/foot}%
    \hbox to .5\paperwidth{\hfil\insertsectionhead\hfil}
  \end{beamercolorbox}%
  \begin{beamercolorbox}[wd=.5\paperwidth,ht=2.5ex,dp=1.125ex]{section in head/foot}%
    \hbox to .5\paperwidth{\hfil\insertframenumber/\inserttotalframenumber\hfil}
  \end{beamercolorbox}%
}


\setbeamertemplate{footline}
{
  \leavevmode%
  \hbox{%
  \begin{beamercolorbox}[wd=.333333\paperwidth,ht=2.25ex,dp=1ex,center]{section in head/foot}%
    \usebeamerfont{title like}\insertinstitute
  \end{beamercolorbox}%
  \begin{beamercolorbox}[wd=.333333\paperwidth,ht=2.25ex,dp=1ex,center]{section in head/foot}%
    \usebeamerfont{title in head/foot}\insertsubsection
  \end{beamercolorbox}%
  \begin{beamercolorbox}[wd=.333333\paperwidth,ht=2.25ex,dp=1ex,right]{section in head/foot}%
    \usebeamerfont{date in head/foot}\insertauthor \hspace*{2ex} 
  \end{beamercolorbox}}%
  \vskip0pt%
}

\setbeamertemplate{navigation symbols}{}

\begin{document}

  \begin{frame}[noframenumbering, plain, c]\frametitle{}
    \maketitle
  \end{frame}

  \begin{frame}[c]\frametitle{Why should we investigate this topic?}
      
    Many physical phenomena are associated with strongly non uniform situations \footnote{i.e. situations for which the avg. number density exhibits spatial variation} and DFT is capable of a good description of this phenomena.
  
  \end{frame}

  \note{a good example of this phenomena is the liquid vapor phase transition}
  
  \section{Review of the framework} % (fold)
  \label{sec:review_of_the_framework}
  \begin{frame}[c]\frametitle{Hamiltonian and Grand potential}
    The hamiltonian is:
    \begin{equation}
      \begin{split}
      H_N &= \sum_i^N \frac{\mathbf{p}_i^2}{2 m} + \Phi(\mathbf{r}_1\dots \mathbf{r}_n) + \sum_{i=1}^N V(\mathbf{r}_i)\\
       &= K.E. + \Phi +V
      \end{split}
    \end{equation}
    And the Grand potential is:
    \begin{equation*}
      \Omega(V, T, \mu,[V(\mathbf{r})]) 
    \end{equation*}
    by defining the useful quantity:
    \begin{equation}
      u(\mathbf{r})\equiv\mu - V(\mathbf{r})
    \end{equation}
    we can claim that:
    \begin{equation}
       \Omega(V, T, \mu,[V(\mathbf{r})]) \rightarrow \Omega(V, T, [u(\mathbf{r})])
    \end{equation}
  \end{frame}

  \begin{frame}[c]\frametitle{First set of correlation functions}
    It is now possible to derive a first set of correlation functions.\\
    one body density:
      \begin{equation}
        \rho(\mathbf{r})\equiv\rho^{(1)}{(\mathbf{r})}\equiv\left<\hat\rho(\mathbf{r})\right> = 
        - \frac{\delta\Omega}{\delta u (\mathbf{r})}
      \end{equation}
    density-density correlation function:
    \begin{equation}
      \begin{split}
      G(\mathbf{r}_1,\mathbf{r}_2)&=
      \left<
        \left(
          \hat\rho\left(\mathbf{r}_1\right) - \rho\left(\mathbf{r}_1\right)
        \right)
        \left(
          \hat\rho\left(\mathbf{r}_2\right) - \rho\left(\mathbf{r}_2\right)
        \right)
        \right> \\
      &=-\beta^{-1}\frac{\delta^2\Omega}{\delta u (\mathbf{r}_1)\delta u (\mathbf{r}_2)}
      \end{split}
    \end{equation}
    \only<2>{
    which is linked to the usual 2 body distribution function via:
    \begin{equation}
      G(\mathbf{r}_1,\mathbf{r}_2) = \rho^{(2)}(\mathbf{r}_1,\mathbf{r}_2)-
      \rho(\mathbf{r}_1)\rho(\mathbf{r}_2) + \rho(\mathbf{r_1})\delta(\mathbf{r}_1- \mathbf{r}_2)
    \end{equation}
    }
    \only<3->{and so on.....}
  \end{frame}

  \begin{frame}[c]\frametitle{H-K-M theorem}
      
  We can move our description from $u(\mathbf{r})$ to $\rho(\mathbf{r})$:
  \only<1>{
  \begin{theorem}
    for a given $\Phi$, $\mu$, and $T$: $\rho(\mathbf{r}) \leftrightarrow V(\mathbf{r})$.\\
    So a density is associated to one external potential and vice versa.
  \end{theorem}
  }
  \begin{equation}
    \begin{matrix}
      & & & & \rightarrow & f_n \text{ is fixed} \\
      \rho(\mathbf{r}) \text{is fixed} & \rightarrow & V(\mathbf{r}) & \rightarrow H & \\
      & & & & \rightarrow & \Xi \text{ is fixed} 
    \end{matrix}
  \end{equation}
  \only<2->{
  where:
  \begin{align}
    f_n &= \Xi^{-1} e^{-\beta H_N - \mu N} \text{ probability density}\\
    \Xi &= Tr_{cl} e^{H_N - \mu N} \text{ grand partition function} 
  \end{align}
  }
  \end{frame}

  \begin{frame}[c]\frametitle{The \textbf{fundamental equation}}
    Following the Mermin idea we introduce the functional:
      \begin{equation}
        \mathcal{F}[\rho] \equiv Tr_{cl}[f_n(K.E. + U + \beta^{-1} ln(f_n))]
      \end{equation}
    \only<1>{
    applying a Legendre transform with respect to $\rho$:
      \begin{equation*}
        \Omega_v[\tilde{\rho}] = \mathcal{F}[\tilde{\rho}] - \int d \mathbf{r} u(\mathbf{r})\tilde{\rho}
      \end{equation*}
    it can be shown that on this quantity there is a variational principle and so:
    \begin{columns}
      \column{.5\textwidth}
      \begin{equation*}
        \left.\frac{\delta \Omega_v[\tilde{\rho}]}{\delta \tilde{\rho}(\mathbf{r})}\right\rvert_{\rho} = 0 
      \end{equation*}
      \column{.5\textwidth}
      \begin{equation*}
        \Omega_v[\rho]= \Omega
      \end{equation*}
    \end{columns}
    combining these conditions with the previous equation we get the fundamental equation.
    }
    \only<2->{
    combining these conditions with the previous equation we get the \textbf{fundamental equation}.
      \begin{equation}
        \mu = V(\mathbf{r}) + \frac{\delta \mathcal{F}[\rho]}{\delta\rho(\mathbf{r})}
      \end{equation}
    that describes the chemical potential in our liquid
    \begin{equation}
      \frac{\delta \mathcal{F}[\rho]}{\delta\rho(\mathbf{r})} \equiv \text{intrinsic chemical potential (non local)}
    \end{equation}
    }  
  \end{frame}

  \begin{frame}[c]\frametitle{Exchange quantity}
  From the definition of $\mathcal{F}$ we can calculate the value for an ideal gas $\Phi = 0$ 
  \begin{equation}
    \mathcal{F}_{id}[\rho] = \beta^{-1} \int d \mathbf{r} \rho(\mathbf{r}) (ln(\Lambda^3 \rho(\mathbf{r})-1))
  \end{equation}
  and we can give a definition of the exchange functional
  \begin{equation}
    \mathcal{F}_{ex}[\rho] = \mathcal{F}[\rho] - \mathcal{F}_{id}[\rho]
  \end{equation}
  \end{frame}

  \begin{frame}[c]\frametitle{Second set of correlation functions}
    From this quantity it is possible to define a second set of correlation functionals
    \begin{equation}
      c^{(1)}= \frac{\delta \beta \mathcal{F}_{ex}[\rho]}{\delta \rho (\mathbf{r})}
    \end{equation}
    \begin{equation}
      c^{(2)}= \frac{\delta^2 \beta \mathcal{F}_{ex}[\rho]}{\delta \rho (\mathbf{r}_1) \delta \rho (\mathbf{r}_2)}
  \end{equation}
  and so on ...
  \end{frame}

  \begin{frame}[c]\frametitle{Self consistent equation}
  Now, using:

  \begin{columns}
    \column{.5\textwidth}
    \begin{equation*}
      c^{(1)}= \frac{\delta \beta \mathcal{F}_{ex}[\rho]}{\delta \rho (\mathbf{r})}
    \end{equation*}
    \column{.5\textwidth}
    \begin{equation*}
      \mu = V(\mathbf{r}) + \frac{\delta \mathcal{F}[\rho]}{\delta\rho(\mathbf{r})}
    \end{equation*}
  \end{columns}
  
  \begin{equation*}
    \mathcal{F}_{id}[\rho] = \beta^{-1} \int d \mathbf{r} \rho(\mathbf{r})
    (ln(\Lambda^3 \rho(\mathbf{r})-1))
  \end{equation*}

  we can write the self consistency equation:
  \begin{equation}
    \Lambda^3 \rho(\mathbf{r}) = exp \left[\beta u(\mathbf{r}) + c^{(1)}(\mathbf{r})\right]
  \end{equation} 
  \end{frame}

  \begin{frame}[c]\frametitle{Comparison with QM DFT}
  \begin{equation*}
    \Lambda^3 \rho(\mathbf{r}) = exp \left[\beta u(\mathbf{r}) + c^{(1)}(\mathbf{r})\right]
  \end{equation*} 
    \begin{align*}
      \mathcal{F}_{ex}[\rho] &\leftrightarrow \frac{1}{2} \int \int d \mathbf{r} d \mathbf{r'}\frac{n(\mathbf{r})n(\mathbf{r'})}{\mid \mathbf{r} - \mathbf{r'} \mid} + E_{xc}[n]\\
      -\beta c^{(1)}(\mathbf{r'}) &\leftrightarrow \int d \mathbf{r} \frac{n(\mathbf{r'})}{\mid \mathbf{r} - \mathbf{r'} \mid} + \frac{\delta E_{xc}[n]}{\delta n}
    \end{align*}
  \end{frame}
  % section review_of_the_framework (end)

  \section{Towards LDA} % (fold)
  \label{sec:towards_lda}
  \begin{frame}[c]\frametitle{Approximation}
    Up to here there were no approximations or restrictions on the problem. But this reflects on the fact that no one told us how to write $ c^{(1)}(\mathbf{r})$.\\

    Now we are going to add them.
  \end{frame}

  \begin{frame}[c]\frametitle{Pairwise potential}
      
  \begin{equation}
    \Phi(\mathbf{r}_1\dots \mathbf{r}_N)=\frac{1}{2}\sum_{i\ne j} \Phi(\mathbf{r}_i,\mathbf{r}_j)
  \end{equation}
    due to the fact that now $\Omega[\Phi]$ we can derive it
  % and since it holds that:
  %\begin{equation}
  %  \frac{1}{2} \rho^2(\mathbf{r}_1 \mathbf{r}_2) = 
  %  \left<\hat{\rho}(\mathbf{r}_1)\hat{\rho}(\mathbf{r}_2)\right> - 
  %  \left<\hat{\rho}(\mathbf{r}_1)\right> \delta(\mathbf{r}_1 - \mathbf{r}_2)
  %\end{equation}
  \begin{equation}
    \frac{\delta \Omega}{\delta \Phi(\mathbf{r}_i,\mathbf{r}_j)} = \frac{1}{2} \rho^2(\mathbf{r}_1 \mathbf{r}_2) = \frac{\delta \Omega_v[\rho]}{\delta \Phi(\mathbf{r}_i,\mathbf{r}_j)}
  \end{equation}
  and using the definition of $\Omega_v$ we have that:
  \begin{equation}
    \frac{\delta \mathcal{F}[\rho]}{\delta \Phi(\mathbf{r}_i,\mathbf{r}_j)} = \frac{1}{2} \rho^2(\mathbf{r}_1 \mathbf{r}_2)
  \end{equation}
  \end{frame}

  \begin{frame}[c]\frametitle{Pairwise potential}
    Parameterizing with alpha a path in the potential space from a reference to the real one ($\Phi = \Phi_r + \alpha \Phi_p$) we can now integrate the last equation:
    \begin{equation}
      \mathcal{F}[\rho] = \mathcal{F}_r[\rho]+\frac{1}{2} \int_0^1 d \alpha \int d \mathbf{r}_1 \int d\mathbf{r}_2 \rho^2(\Phi_\alpha; \mathbf{r}_1, \mathbf{r}_2) \Phi_p(\mathbf{r}_1,\mathbf{r}_2)
    \end{equation}
    where $\mathcal{F}_r[\rho]$ is the intrinsic free energy of the reference fluid.\\
    \smallskip
    This equation is the starting point for all the perturbations theory.
  \end{frame}

  \begin{frame}[c]\frametitle{Some remark}
      \begin{itemize}
        \item $\mathcal{F}[\rho]$ is a functional that is unknown and that we have to built.
        \item after $\mathcal{F}[\rho]$ has been built there is no reason to expect the resulting proprieties coming from it to be the exact solution of any hamiltonian.
        \item DFT is no good to describe phase transition because $\mathcal{F}[\rho]$ usually has a mean field character
      \end{itemize}
  \end{frame} 
  % section towards_lda (end)

  \section{LDA} % (fold)
  \label{sec:lda}
  \begin{frame}[c]\frametitle{LDA}
  If we are dealing with a perturbation potential we can expand our last result:
  \begin{equation}
    \mathcal{F}[\rho] = \mathcal{F}_r[\rho]+\frac{1}{2} \int d \mathbf{r}_1 \int d\mathbf{r}_2 \rho^2_r(\mathbf{r}_1, \mathbf{r}_2) \Phi_p(\mathbf{r}_1,\mathbf{r}_2)+ O(\Phi_p^2)
  \end{equation}
  were now we can plug our big approximation: the ``local density''
  \begin{align*} 
    \mathcal{F}_r[\rho] & \approx \int d \mathbf{r} f_{r0}(\rho(\mathbf{r})) \\
     \rho^2_r(\mathbf{r}_1, \mathbf{r}_2) & \approx \rho(\mathbf{r}_1) \rho(\mathbf{r}_2) g_r(\bar{\rho}; r_{12})
  \end{align*} 
  where:
  \begin{columns}
    \column{.7\textwidth}
    $f_{r0}(\rho(\mathbf{r}))$: fed of the reference fluid\\
    $g_r(\bar{\rho}; r_{12})$: rdf of the reference fluid
    \column{.3\textwidth}
    e.g. $\bar{\rho}=\frac{\rho(\mathbf{r_1})+\rho(\mathbf{r_2})}{2}$
  \end{columns}
  \end{frame}
  
  \begin{frame}[c]\frametitle{How to chose the reference fluid}
    \only<1>{
    \begin{figure}[c]
      \centering
      \includegraphics[width=.9\textwidth]{graph/view1.eps}
    \end{figure}
    }
    \only<2>{
    \begin{equation*}
      v(r)= 4\left[\left(\frac{\sigma}{r}\right)^{12}-\left(\frac{\sigma}{r}\right)^{6}\right]
    \end{equation*}
    \begin{columns}
      \column{.5\textwidth}
      Barker and Henderson
      \begin{align*}
      v_0(r)&= \begin{cases}
              v(r) & r<\sigma \\
              0 & r>\sigma 
            \end{cases}\\
      w(r)&= \begin{cases}
              0 & r<\sigma\\ 
              v(r) & r>\sigma
            \end{cases}
      \end{align*}
      \column{.5\textwidth}
      Weeks et al.
      \begin{align*}
      v_0(r)&= \begin{cases}
              v(r) + \epsilon & r<r_m \\
              0 & r>r_m 
            \end{cases}\\
      w(r)&= \begin{cases}
              -\epsilon & r<r_m\\ 
              v(r) & r>r_m
            \end{cases}
      \end{align*}
    \end{columns}
    }
    \par\hfill \tiny $v_0(r)$ is the reference
  \end{frame}

  \begin{frame}[c]\frametitle{Example of results}
  Surface tension of Argon near the triple point\footnote{$\sigma=$\SI{3.40}{\angstrom}, $\frac{\epsilon}{k_b}=$\SI{119.8}{\kelvin}}
    \begin{table}[c]
      \centering
      \begin{tabular}{c|l|cc}
      \hline

      \hline
      \multicolumn{2}{c|}{Theory}  & T(K) & $\gamma$(dyn cm\^-1) \\
      \hline
        &Experimental result & 85 & 13.1 \\
        &Experimental result & 90 & 11.9 \\
        BH &Toxvaerd, S. 1971 & 90 & 14 \\
        BH &Lee, J.K. 1974 & 84 & 16 \\
        BH &Abraham, F.F. 1975 & 90 & 17 \\
        WCA &Abraham, F.F. 1975 & 90 & 17 \\
        WCA &Upstill, C.E. 1977 & 85 & 14 \\
        WCA &Singh, Y. 1977 & 84 & 18 \\
      \hline
      
      \hline
      \end{tabular}
    \end{table}
  \end{frame}

  \begin{frame}[c]\frametitle{Conclusions}
    \begin{itemize}
      \item we saw how to get to a Self Consistency equation
      \item we explored a perturbation approach to define $\mathcal{F}[\rho]$
      \item we applied an LDA to this result and show some result
    \end{itemize}
    \pause
    possible future?
    \begin{itemize}
      \item investigate different approaches for $\mathcal{F}[\rho]$
      \item use them in non local theory
    \end{itemize}
  
  \end{frame}
  \begin{frame}[c]\frametitle{Conclusions}
  Main sources that have been used for this work:
  \begin{itemize}
    \item The nature of the liquid-vapor interface and other topics by R.Evans 1978
    \item Fundamentals of Inhomogeneous Fluids by Douglas Henderson 1992
    \item Theory of simple liquids by J.P. Hansen and I.R. McDonald 1976  
    \item Statistical Mechanical Theories of Freezing by M.Baus 1987
  \end{itemize}
  \begin{center}
    Thank you for your attention
  \end{center}
  \end{frame}
  % section lda (end)

\end{document}